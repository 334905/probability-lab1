\documentclass{article}

\usepackage{ifluatex}
\ifluatex
\usepackage{fontspec}
\setsansfont{CMU Sans Serif}%{Arial}
\setmainfont{CMU Serif}%{Times New Roman}
\setmonofont{CMU Typewriter Text}%{Consolas}
\defaultfontfeatures{Ligatures={TeX}}
\else
\errmessage{Run with LuaLaTeX}
\fi

\usepackage[english,russian]{babel}
\usepackage{amssymb,latexsym,amsmath,amscd,mathtools,wasysym}
\usepackage[shortlabels]{enumitem}
\usepackage[makeroom]{cancel}
\usepackage{graphicx}
\usepackage{geometry}
\usepackage{verbatim}
\usepackage{fvextra}

\usepackage{longtable}
\usepackage{multirow}
\usepackage{multicol}
\usepackage{tabu}
\usepackage{arydshln} % \hdashline and :

\usepackage{float}
\makeatletter
\g@addto@macro\@floatboxreset\centering
\makeatother
\usepackage{caption}
\usepackage{csquotes}
\usepackage[bb=dsserif]{mathalpha}
\usepackage[normalem]{ulem}

\usepackage{xcolor}

\DeclareFontFamily{U}{matha}{\hyphenchar\font45}
\DeclareFontShape{U}{matha}{m}{n}{
    <5> <6> <7> <8> <9> <10> gen * matha
    <10.95> matha10 <12> <14.4> <17.28> <20.74> <24.88> matha12
}{}
\DeclareSymbolFont{matha}{U}{matha}{m}{n}
\DeclareFontFamily{U}{mathb}{\hyphenchar\font45}
\DeclareFontShape{U}{mathb}{m}{n}{
    <5> <6> <7> <8> <9> <10> gen * mathb
    <10.95> matha10 <12> <14.4> <17.28> <20.74> <24.88> mathb12
}{}
\DeclareSymbolFont{mathb}{U}{mathb}{m}{n}

\DeclareMathSymbol{\tmp}{\mathrel}{mathb}{"15}
\let\defeq\tmp
\DeclareMathSymbol{\tmp}{\mathrel}{mathb}{"16}
\let\eqdef\tmp

\usepackage{hyperref}
\hypersetup{
    %hidelinks,
    colorlinks=true,
    linkcolor=darkgreen,
    urlcolor=blue,
    breaklinks=true,
}

\usepackage{pgf}
\usepackage{pgfplots}
\pgfplotsset{compat=newest}
\usepackage{svg}
\usepackage{tikz,tikz-3dplot}
\usepackage{tkz-euclide}
\usetikzlibrary{calc,automata,patterns,angles,quotes,backgrounds,shapes.geometric,trees,positioning,decorations.pathreplacing}
\pgfkeys{/pgf/plot/gnuplot call={T: && cd TeX && gnuplot}}
\usepgfplotslibrary{fillbetween,polar}
\usetikzlibrary{quotes,babel}
\ifluatex
\usetikzlibrary{graphs,graphs.standard,graphdrawing}
\usegdlibrary{layered,trees,circular,force}
\fi
\makeatletter
\newcommand\currentnode{\the\tikz@lastxsaved,\the\tikz@lastysaved}
\makeatother

\usepackage{adjustbox}

\geometry{margin=1in}
\usepackage{fancyhdr}
\pagestyle{fancy}
\fancyfoot[L]{}
\fancyfoot[C]{Иванов Тимофей}
\fancyfoot[R]{\pagename\ \thepage}
\fancyhead[R]{\leftmark}
\fancyhead[L]{Теория вероятности}
\fancyhead[C]{Лабораторная работа 1}
\renewcommand{\sectionmark}[1]{\markboth{#1}{}}

\makeatletter
\def\@seccntformat#1{%
    \expandafter\ifx\csname c@#1\endcsname\c@section\else
    \csname the#1\endcsname\quad
    \fi}
\makeatother

\usepackage{asymptote}

\begin{document}
    \begin{asydef}
        settings.render = -1;
        import three;
        import graph3;
    \end{asydef}
    \section{Задача 1}
    \paragraph{Условие.}
    В городе с населением в $n+1$ человек некто узнаёт новость. Он передаёт её первому встречному, тот~--- ещё одному и т.д. На каждом шагу впервые узнавший новость может сообщить её любому из $n$ человек  с одинаковыми вероятностями.\\
    Найти вероятность того, что в продолжение $r$ единиц времени
    \begin{enumerate}
        \item Новость не возвратится к человеку, который узнал её первым.
        \item Новость не будет никем повторена.
    \end{enumerate}
    Решить ту же задачу в предположении, что на каждом шагу новость сообщается группе из $N$ случайно выбранных людей.
    \paragraph{Решение.}
    В первом случае задачи решение довольно просто, а для $N\neq1$ решить я её не могу.\\
    В случае $N=1$ новость в любой момент времени передаёт не более 1 человека (1, если он получил её в прошлый момент времени впервые, 0, если новость пришла к тому, кто её уже знал).
    \subparagraph{a.}
    Посчитаем вероятность, что новость вернётся к первому человеку.\\
    В первый момент времени второй человек узнаёт новость. Потом он с вероятностью $\frac1n$ говорит её первому, на чём всё заканчивается. В противном случае он передаёт новость третьему... $k$-тый человек имеет вероятность $\frac1n$ передать её первому, $\frac{k-2}n$~--- передать её не-первому человеку, который новость уже знает и $\frac{n-k+1}n$~--- передать новость новому ($k+1$-му человеку). Итого вероятность того, что новость вернётся к первому, составляет
    $$
    \frac1n+\frac{n-1}n\left(\frac1n+\frac{n-2}n\left(\frac1n+\frac{n-3}n\left(\frac1n+\cdots\frac{n-r+2}n\left(\frac1n\right)\cdots\right)\right)\right)
    $$
    Это можно упростить до:
    $$
    \frac1n+\frac{n-1}n\frac1n+\frac{(n-1)(n-2)}{n^2}\frac1n+\cdots+\frac{(n-1)(n-2)\cdots(n-r+2)}{n^{r-2}}\frac1n
    $$
    Как это просуммировать, правда, я не знаю. Ответом является разность единицы и этой величины.
    \subparagraph{б.}
    Не очень понятно, что имеется в виду под <<повторена>>, если новость сообщает только впервые её услышавший. Вероятно, имеется в виду, что никто не услышит новость дважды. Тогда нам подходит ситуация, когда $k$-тый человек передаёт новость любому из $n-k$ не слышавших её, то есть искомая вероятность равна
    $$
    \frac{n-1}n\cdot\frac{n-2}n\cdot\cdots\cdot\frac{n-r+1}n
    $$
    Как это упростить, я всё ещё понятия не имею.
    \section{Задача 2.}
    \paragraph{Условие.}
    Случайная точка $A$ имеет равномерное распределение в правильном $n$-угольнике. Найти вероятность $P_n$, что точка $A$ находится ближе к границе многоугольника, чем к его диагоналям. Найти числа $C$, $\alpha$, что
    $$
    P_n=Cn^\alpha(1+o(1))
    $$
    \paragraph{Решение.}
    \begin{figure}[H]
        \includesvg[scale=.5]{task2}
    \end{figure}
    Пусть $A_1\cdots A_n$~--- искомый многоугольник. Нам нужно посчитать площадь той части, где точки ближе к сторонам, чем к диагоналям. Несложно заметить, что граница, разделяющая точки, которые ближе к одной прямой, чем к другой~--- биссектриса угла между ними. Т.е., если обратить внимание на рисунок выше, точки, которые ближе к $A_2A_3$, чем к $A_1A_3$ находятся <<ниже>> биссектрисы угла $A_1A_3A_2$ (т.е. <<ниже>> прямой $A_3C_{23}$).\\
    Несложно заметить, что в треугольнике $A_2C_{23}A_3$ находятся точки, которые ближе к $A_2A_3$, чем к \textbf{любой} из диагоналей. И нигде в другом месте такие точки не находятся. То есть всё, что нам остаётся,~--- найти площадь этого треугольника, умножить её на $n$ (потому что около каждой стороны есть такой) и поделить полученное на площадь многоугольника.\\
    Пусть сторона многоугольника равна $1$. Его площадь тогда равна
    $$
    \frac n4\cot\frac\pi n
    $$
    Теперь давайте посчитаем площадь треугольника $A_2C_{23}A_3$. Он, как несложно заметить, равнобедренный, а его основание~--- $1$. Если посчитать углы, с площадью можно будет справиться.\\
    Рассмотрим $\triangle A_1A_2A_3$. Он равнобедренный и в нём $\angle A_1A_2A_3=\frac{\pi(n-2)}n$, а значит $\angle A_2A_3A_1=\angle A_3A_1A_2=\frac\pi n$. Следовательно $\angle C_{23}A_3A_2=\frac{\pi}{2n}$, и аналогично $\angle C_{23}A_2A_3=\frac{\pi}{2n}$. А отсюда $\angle A_2C_{23}C_3=\frac{\pi(n-1)}n$. По формуле площади треугольника через три угла и сторону
    $$
    S_{\triangle A_2C_{23}A_3}=\frac{\left(\sin\frac{\pi}{2n}\right)^2}{2\sin\frac{\pi(n-1)}n}
    $$
    Итого ответом к задаче является
    $$
    \frac{n\frac{\left(\sin\frac{\pi}{2n}\right)^2}{2\sin\frac{\pi(n-1)}n}}{\frac n4\cot\frac\pi n}=\frac{2\left(\sin\frac{\pi}{2n}\right)^2}{\sin\frac{\pi(n-1)}n\cot\frac\pi n}=\frac{1-\cos\frac\pi n}{\sin\left(\pi-\frac\pi n\right)\cot\frac\pi n}=\frac{1-\cos\frac\pi n}{\cos\frac\pi n}=\frac1{\cos\frac\pi n}-1
    $$
    Осталось только оценить $P_n$:
    \[\begin{split}
        \lim\limits_{x\to\infty}\frac{\frac1{\cos\frac\pi x}-1}{Cx^\alpha}&=1\overset{\hat o}\Leftrightarrow
        \lim\limits_{x\to\infty}\frac{-\frac{\pi\tan\frac\pi x}{x^2\cos\frac\pi x}}{C\alpha x^{\alpha-1}}=1\Leftrightarrow
        \lim\limits_{x\to\infty}\frac{-\frac{\pi\tan\frac\pi x}{\cos\frac\pi x}}{C\alpha x^{\alpha+1}}=1\overset{\hat o}\Leftrightarrow\\
        &\overset{\hat o}\Leftrightarrow
        \lim\limits_{x\to\infty}\frac{\frac{\pi^2\left(\tan^2\frac\pi x+\frac1{\cos^2\frac\pi x}\right)}{x^2\cos\frac\pi x}}{C\alpha(\alpha+1) x^{\alpha}}=1\Leftrightarrow
        \lim\limits_{x\to\infty}\frac{\frac{\pi^2\left(\tan^2\frac\pi x+\frac1{\cos^2\frac\pi x}\right)}{\cos\frac\pi x}}{C\alpha(\alpha+1) x^{\alpha+2}}=1\Leftrightarrow\\
        &\Leftrightarrow
        \left\{\begin{aligned}
            &\frac{\pi^2}{C\alpha(\alpha+1)}=1\\&\alpha+2=0
        \end{aligned}\right.
        \Leftrightarrow
        \left\{\begin{aligned}
            &C=\frac{\pi^2}2\\&\alpha=-2
        \end{aligned}\right.
    \end{split}\]
    Ответ: $P_n=\frac{\pi^2}2n^{-2}(1+o(1))$.
    \section{Задача 3.}
    \paragraph{Условие.}
    Введем события $A_i=\{X=i\}$, $B_i=\{Y=i\}$, $i\geqslant0$. Известно, что для любых $i\geqslant0$ и $j\geqslant0$ события $A_i$ и $B_i$ независимы и
    $$
    P(X=i)=e^{-\lambda}\frac{\lambda^i}{i!}\qquad \lambda>0
    $$
    $$
    P(Y=j)=e^{-\mu}\frac{\mu^i}{i!}\qquad \mu>0
    $$
    Найти $P(X=i\mid X+Y=j)$.
    \paragraph{Трактовка условия.}
    Для начала давайте поймём, что такое $i$ и $j$, исходя из этого условия. На мой взгляд, это неотрицательное \textbf{целое} число т.к.
    $$
    \sum\limits_{i=0}^\infty P(X=i)=e^{-\lambda}\underbrace{\sum\limits_{i=0}^\infty \frac{\lambda^i}{i!}}_{\text{Ряд Тейлора }e^{\lambda}}=1
    $$
    \paragraph{Решение.}
    По определению условной вероятности
    $$
    P(X=i\mid X+Y=j)=\frac{P(X+Y=j\land X=i)}{P(X+Y=j)}=\frac{P(Y=j-i\land X=i)}{P(X+Y=j)}
    $$
    Отсюда сразу видно, что если $j<i$ или $i<0$, то искомая условная вероятность~--- ноль, а если $j<0$, то не определена. Числитель этой дроби понятно какой, а вот знаменатель надо посчитать. Зная, что $i$ и $j$, целые (и неотрицательные), разобьём $\{X+Y=j\}$ на следующие попарно несовместные события:
    \begin{enumerate}
        \addtocounter{enumi}{-1}
        \item $\{X=0\land Y=j\}$.
        \item $\{X=1\land Y=j-1\}$.
        \item[...]
        \item[$j$.] $\{X=j\land Y=0\}$.
    \end{enumerate}
    Вероятности их соотвественно равны
    \begin{enumerate}
        \addtocounter{enumi}{-1}
        \item $$e^{-\lambda}\frac{\lambda^0}{0!}e^{-\mu}\frac{\mu^j}{j!}$$
        \item $$e^{-\lambda}\frac{\lambda^1}{1!}e^{-\mu}\frac{\mu^{j-1}}{(j-1)!}$$
        \item[...]
        \item[$j$.] $$e^{-\lambda}\frac{\lambda^j}{j!}e^{-\mu}\frac{\mu^0}{0!}$$
    \end{enumerate}
    Поскольку эти события несовместны, а их объединение равно $\{X+Y=j\}$, надо лишь сложить искомые вероятности.
    $$
    \sum\limits_{i=0}^je^{-\lambda}\frac{\lambda^i}{i!}e^{-\mu}\frac{\mu^{j-i}}{(j-i)!}=\frac{e^{-\lambda-\mu}}{j!}\sum\limits_{i=0}^j\frac{j!}{i!(j-i)!}\lambda^i\mu^{j-i}=\frac{(\lambda+\mu)^j}{j!\cdot e^{\lambda+\mu}}
    $$
    Осталось лись поделить $P(X=i\land Y=j-i)$ на это.\\
    Ответ: $\displaystyle\binom ji\frac{\lambda^i\mu^{j-i}}{(\lambda+\mu)^j}$.
    \section{Задача 4.}
    \paragraph{Условие.}
    Рассмотрите схемы Бернулли при $n\in\{10,100,1000,10000\}$ и $p\in\{0.001,0.01,0.1,0.25,0.5\}$ и рассчитайте точные вероятности (где это возможно) $P\left(S_n\in\left[\frac n2-\sqrt{npq};\frac n2+\sqrt{npq}\right]\right)$, $S_n$~--- количество успехов в $n$ испытаниях, и приближенную с помощью одной из предельных теорем. Сравните точные и приближенные вероятности.\\
    Объясните результаты.
    \paragraph{Решение.}
    Для начала, это очень просто оценить <<\textit{с помощью одной из предельных теорем}>>. Согласно интегральной теореме Муавра~--- Лапласа,
    $$
    P\left(x_1\sqrt{npq}+np\leqslant S_n\leqslant x_2\sqrt{npq}+np\right)\approx\frac1{\sqrt{2\pi}}\int\limits_{x_1}^{x_2}e^{-\frac{t^2}2}~\mathrm dt
    $$
    Несложно вывести формулу
    $$
    x_1=-1+\frac{(1-2p)\sqrt{npq}}{2p(1-p)}\qquad x_2=1+\frac{(1-2p)\sqrt{npq}}{2p(1-p)}
    $$
    \begin{figure}[H]
        \begin{tabular}{|c|c|c|c|c|}
            \hline
            $n$ & $p$ & $x_1$ & $x_2$ & $\frac1{\sqrt{2\pi}}\int\limits_{x_1}^{x_2}e^{-\frac{t^2}2}~\mathrm dt$\\
            \hline
            $10$ & $0.001$ & $-1+\frac{499\sqrt{1110}}{333}$ & $1+\frac{499\sqrt{1110}}{333}$ & $5.48\times 10^{-523}$\\
            \hline
            $100$ & $0.001$ & $-1+\frac{4990\sqrt{111}}{333}$ & $1+\frac{4990\sqrt{111}}{333}$ & $8.99\times 10^{-5348}$\\
            \hline
            $1000$ & $0.001$ & $-1+\frac{4990\sqrt{1110}}{333}$ & $1+\frac{4990\sqrt{1110}}{333}$ & $1.27\times 10^{-53911}$\\
            \hline
            $10000$ & $0.001$ & $-1+\frac{49900\sqrt{111}}{333}$ & $1+\frac{49900\sqrt{111}}{333}$ & $2.48\times 10^{-540559}$\\
            \hline
            $10$ & $0.01$ & $-1+\frac{49\sqrt{110}}{33}$ & $1+\frac{49\sqrt{110}}{33}$ & $8.29\times 10^{-49}$\\
            \hline
            $100$ & $0.01$ & $-1+\frac{490\sqrt{11}}{33}$ & $1+\frac{490\sqrt{11}}{33}$ & $1.13\times 10^{-508}$\\
            \hline
            $1000$ & $0.01$ & $-1+\frac{490\sqrt{110}}{33}$ & $1+\frac{490\sqrt{110}}{33}$ & $1.15\times 10^{-5202}$\\
            \hline
            $10000$ & $0.01$ & $-1+\frac{4900\sqrt{11}}{33}$ & $1+\frac{4900\sqrt{11}}{33}$ & $3.02\times 10^{-52454}$\\
            \hline
            $10$ & $0.1$ & $-1+\frac{4\sqrt{10}}{3}$ & $1+\frac{4\sqrt{10}}{3}$ & $2.59\times 10^{-4}$\\
            \hline
            $100$ & $0.1$ & $\frac{37}3$ & $\frac{43}3$ & $1.2\times 10^{-35}$\\
            \hline
            $1000$ & $0.1$ & $-1+\frac{40\sqrt{10}}{3}$ & $1+\frac{40\sqrt{10}}{3}$ & $4.38\times 10^{-371}$\\
            \hline
            $10000$ & $0.1$ & $\frac{397}3$ & $\frac{403}3$ & $2.36\times 10^{-3806}$\\
            \hline
            $10$ & $0.25$ & $-1+\frac{\sqrt{30}}{3}$ & $1+\frac{\sqrt{30}}{3}$ & $8.1\times 10^{-2}$\\
            \hline
            $100$ & $0.25$ & $-1+\frac{10\sqrt{3}}{3}$ & $1+\frac{10\sqrt{3}}{3}$ & $3.61\times 10^{-7}$\\
            \hline
            $1000$ & $0.25$ & $-1+\frac{10\sqrt{30}}{3}$ & $1+\frac{10\sqrt{30}}{3}$ & $1.96\times 10^{-63}$\\
            \hline
            $10000$ & $0.25$ & $-1+\frac{100\sqrt{3}}{3}$ & $1+\frac{100\sqrt{3}}{3}$ & $3.02\times 10^{-702}$\\
            \hline
            Сколько угодно & $0.5$ & $-1$ & $1$ & $2.7\times10^{-1}$\\
            \hline
        \end{tabular}
    \end{figure}\noindent
    Для $n=10$ посчитаем искомые вероятности явно.
    \begin{figure}[H]
        \begin{tabular}{|c|c|c|}
            \hline
            $p$ & Подходящие исходы & Вероятность\\
            \hline
            $0.001$ & $\{5\}$ & $\binom{10}{5}0.001^50.999^5=\frac{63}{250000}\approx 2.5\times 10^{-4}$\\
            \hline
            $0.01$ & $\{5\}$ & $\binom{10}{5}0.01^50.99^5=\text{много цифр}\approx 2.4\times 10^{-2}$\\
            \hline
            $0.1$ & $\{5\}$ & $\binom{10}{5}0.1^50.9^5=\text{много цифр}\approx 1.5\times 10^{-3}$\\
            \hline
            $0.25$ & $\{4;5;6\}$ & $\binom{10}{4}0.25^40.75^6+\binom{10}{5}0.25^50.75^5+\binom{10}{6}0.25^60.75^4=\frac{28917}{131072}\approx 2.2\times 10^{-1}$\\
            \hline
            $0.5$ & $\{4;5;6\}$ & $\binom{10}{4}0.5^40.5^6+\binom{10}{5}0.5^50.5^5+\binom{10}{6}0.5^60.5^4=\frac{21}{32}\approx 6.6\times 10^{-1}$\\
            \hline
        \end{tabular}
    \end{figure}\noindent
    Разумеется, для $n=10$ результаты сходятся с оценкой довольно плохо т.к. она нормально работает только при больших $n$.
\end{document}